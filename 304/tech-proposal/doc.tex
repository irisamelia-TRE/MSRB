\documentclass{article}

\title{Retirement Calculator Project Technology Usage Proposal}
\author{Team 304}


\usepackage{graphicx}
\graphicspath{ {img/} }
\usepackage{float}

\setlength{\parskip}{1em}

\begin{document}

\maketitle

\section*{Frontend Technology}

For the frontend of the calculator, we plan to use angular js. 
Angular js is a javascript framework which excels at creating
static documents. We chose Angular js for two main reasons:
the first is that it is easily extendable and the second is 
that it will handle a lot of the workload for us.

At a basic level, we will be using Angular js to create a
form and a results page. However, the form can have several
different views depending on what the user wants to find out.
Using angular js, we can easily handle that rendering logic.

In later version of our product, we want to be able to pull
some, if not all, the data from a pre-existing database. In
order to do this, we need to make sure the user is 
authenticated. However, since there's also a non-authenticated
workflow for this site, we need to be able to differentiate
and handle both flows. Using Angular js, we can easily handle
authorized and non-authorized workflows via the secure and
non-secure routes.

We will send and receive data from an exposed API created 
by our backend.

\section*{Backend Technology}

In order to implement the backend to the calculator
we plan to use Java Enterprise Edition (Java EE).
Java EE is an exceedingly mature technology,
with a wealth of resources and compatible software available.
Furthermore,
in this case specifically,
preexisting technology related to the course is set up for Java
which will reduce the overhead from technology choice.
That preexisting technology is Jenkins and JBoss,
a CI solution and an application server.
We will be using Jenkins to manage builds and testing,
and JBoss to host our application.

The backend project will be structured as some number of Java servlets,
the business logic,
and code to connect to the database.
To connect to the database we will use a JDBC driver for PostgreSQL.
We will use Maven to build the project.

\section*{Database}

In order to provide the database needed for development of the calculator
we will use PostgreSQL on AWS.
These technologies are all mature and very powerful,
and specifically chosen (from among similar technologies)
because of previous experience on the team in using them.
That previous experience will allow us
to deliver more value to the client
(as dicussed in the Backend Technology section)
by minimizing startup costs.

\end{document}

%%% Local Variables:
%%% mode: latex
%%% TeX-master: t
%%% End:
